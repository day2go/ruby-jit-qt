\section{TODO列表}


\section{默认参数值的传递表达}
\subsection{普通类型}
这种类型包括内置类型,int,char,QVariant支持的类。
这种可以直接使用QVariant表示,并且能够非常容易的检测。

\subsection{表达式参数值}
这种包括类似QChar(' '), QString(""),QFlags<???>(),
在FrontEngine中转换打包这个表达式,
传递到CompilerEngine进行编译相应的表达式,并传递到OperatorEngine中引用表达式结果。
表达式的结果应该在CompilerEngine中生成,因为底层不应该再调用上层。
不过这个表达式代码还是挺难生成的。

默认值表达式生成的执行逻辑优化:
1、目前是在FE中,检测表达式中的类类型,在AST查找时就明确生成这个类的实例,把这个类传递到IR生成类OE中使用。
这种实现没有灵活性,并且需要在后续的测试中不断加入新的默认值类型的处理。
2、现在,已经使用了更优化的方式,为默认值表达式生成一段IR代码,传递到IR生成类OE中使用。
一般默认值表达式不会太复杂,生成的IR代码也比较简单,也容易处理使用。
3、也许,后续还可以把这些默认值表达式缓存,把生成的IR代码也缓存,不需要每次使用的时候都重复处理了。

// 发现了点东西,拷贝IR指令时需要注意的。
/*
  可能就是api中说的,clone出来的Inst与原来的Inst有点不一样,没有Parent
  以下两名都能生成相同的一行指令,但前者生成的语句就会导致崩溃,后者生成的则无问题
  call void @_ZN6QFlagsIN2Qt10WindowTypeEEC2EMNS2_7PrivateEi(%class.QFlags* %toargx0, i32 -1)
  call void @_ZN6QFlagsIN2Qt10WindowTypeEEC2EMNS2_7PrivateEi(%class.QFlags* %toargx0, i32 -1)
*/


\subsection{void*类型}
这种是一个透明传递的QXxx*类型实例。

\section{模板方法的实例化}
\subsection{普通方法实例化}
\subsection{构造函数实例化}
默认情况下会生成Ctor\_Complete类型的实例。

\section{预编译头文件AST的应用}
可用。
不完善,修改了其中一个头文件,则需要重新生成ast文件。
如果实时生成,则速度太慢,上秒级。
可以考虑使用增量式的AST生成。


\section{参考}
ROOT/cling
llvm/vmkit
Graal/Truffle

